\documentclass[12pt]{report}
\usepackage[a4paper, total={7.3in, 9.7in}]{geometry}
\usepackage{amsmath}
\usepackage{upquote}
\usepackage{listings}
\usepackage{xcolor}
\usepackage{titlesec}
\usepackage{amssymb}

\definecolor{backgroundcolor}{rgb}{1, 1, 1}
\definecolor{commentstyle}{rgb}{0.365, 0.422, 0.475}
\definecolor{keywordstyle}{rgb}{0.6, 0.14, 0.576}
\definecolor{numberstyle}{rgb}{0.5, 0.5, 0.5}
\definecolor{stringstyle}{rgb}{0.77, 0.1, 0.08}

\lstdefinestyle{xcodecolor}{
    backgroundcolor=\color{backgroundcolor},   
    commentstyle=\color{commentstyle},
    keywordstyle=\color{keywordstyle},
    numberstyle=\scriptsize\color{numberstyle},
    stringstyle=\color{stringstyle},
    basicstyle=\ttfamily\footnotesize,
    breakatwhitespace=false,         
    breaklines=true,                 
    captionpos=b,                    
    keepspaces=true,                   
    numbersep=5pt,                  
    showspaces=false,                
    showstringspaces=false,
    showtabs=false,                  
    tabsize=2
}

\lstset{style=xcodecolor}

\usepackage[T1]{fontenc}
\usepackage{cascadia-code}

% Raised Rule Command:
%  Arg 1 (Optional) - How high to raise the rule
%  Arg 2            - Thickness of the rule
\newcommand{\raisedrule}[2][0em]{\leaders\hbox{\rule[#1]{1pt}{#2}}\hfill}

\setlength{\parindent}{0pt}
\titleformat{\section}
{\normalfont\Large\bfseries}{\thesection}{1em}{}[{\titlerule[0.8pt]}]

\begin{document}

	{\Large
	\textbf{T9}}
	
	\vspace{0.4cm}
	DiPS CodeJam 23\raisedrule[0.25em]{1pt}
	\\
	% document

	\section*{Prompt}
	T9 s a predictive text technology for mobile phones, specifically those that contain a $3 \times 4$ numeric keypad. Letters are typed by pressing corresponding keys repeatedly. For example, pressing $222, 666, 3, 33$ results in the string ``code''.
	Given a sequence of numbers $n[]$, find the string that corresponds to $n$. Assume the following character set:
	
	\begin{tabular}{ c c } 
	 	\textbf{Key} & \textbf{Character}\\
		\hline
		1 & -- \\
		2 & abc \\
		3 & def \\
		4 & ghi \\
		5 & jkl \\
		6 & mno \\
		7 & pqrs \\
		8 & tuv \\
		9 & wxyz \\
		0 & \textvisiblespace \\
	\end{tabular}

	\subsection*{Input Format}
	The first and only line of input contains a space-separated list of presses.

	\subsection*{Output Format}
	Your output should contain one line that contains the resultant string.

	\subsection*{Constraints}
	$1\leq n \leq10^5$

	\subsection*{Sample Input/Output}
	\begin{tabular}{ |l|l| } 
		\hline
		\textbf{Input} & \textbf{Output} \\
		{\lstinputlisting{./testCases/input/input00.txt}} & {\lstinputlisting{./testCases/output/output00.txt}} \\ % use {\lstinputlisting{./testCases/input/input00.txt}} & {\lstinputlisting{./testCases/output/output00.txt}}
		\hline
	\end{tabular}


	\section*{Solution}
	Analysing the input $n[222, 666, 3, 33]$, we can translate the input into characters. For example:
	
	$222$:
	\begin{itemize}
		\item $2$ corresponds to the character set ``abc''.
		\item The digit is repeated 3 times, referring to the 3rd character in the sequence.
		\item Return $\text{``abc''}[2]=\text{c}$
	\end{itemize}


	\section*{Sample Program}
	\lstinputlisting[language=Python]{sampleSolution.py}
	

\end{document}