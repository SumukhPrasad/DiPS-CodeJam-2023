\documentclass[12pt]{report}
\usepackage[a4paper, total={7.3in, 9.7in}]{geometry}
\usepackage{amsmath}
\usepackage{upquote}
\usepackage{listings}
\usepackage{xcolor}
\usepackage{titlesec}
\usepackage{amssymb}

\definecolor{backgroundcolor}{rgb}{1, 1, 1}
\definecolor{commentstyle}{rgb}{0.365, 0.422, 0.475}
\definecolor{keywordstyle}{rgb}{0.6, 0.14, 0.576}
\definecolor{numberstyle}{rgb}{0.5, 0.5, 0.5}
\definecolor{stringstyle}{rgb}{0.77, 0.1, 0.08}

\lstdefinestyle{xcodecolor}{
    backgroundcolor=\color{backgroundcolor},   
    commentstyle=\color{commentstyle},
    keywordstyle=\color{keywordstyle},
    numberstyle=\scriptsize\color{numberstyle},
    stringstyle=\color{stringstyle},
    basicstyle=\ttfamily\footnotesize,
    breakatwhitespace=false,         
    breaklines=true,                 
    captionpos=b,                    
    keepspaces=true,                   
    numbersep=5pt,                  
    showspaces=false,                
    showstringspaces=false,
    showtabs=false,                  
    tabsize=2
}

\lstset{style=xcodecolor}

\usepackage[T1]{fontenc}
\usepackage{cascadia-code}

% Raised Rule Command:
%  Arg 1 (Optional) - How high to raise the rule
%  Arg 2            - Thickness of the rule
\newcommand{\raisedrule}[2][0em]{\leaders\hbox{\rule[#1]{1pt}{#2}}\hfill}

\setlength{\parindent}{0pt}
\titleformat{\section}
{\normalfont\Large\bfseries}{\thesection}{1em}{}[{\titlerule[0.8pt]}]

\begin{document}

	{\Large
	\textbf{Just Missed}}
	
	\vspace{0.4cm}
	DiPS CodeJam 23\raisedrule[0.25em]{1pt}
	\\
	% document

	\section*{Prompt}
	Given a number $n\geq3$, find the the smallest number $k$ such that the modular residues of $k$ by the first $n$ prime numbers belongs to the set $\{-1, 0, 1\}$.
	
	For example, if $n=5$, the first $5$ primes are $2,3,5,7,11$. This means that $k\in[10,\infty)$.
	\begin{itemize}
		\item $10\pmod7=3$, move on to the next number
		\item $11\pmod7=4$, move on to the next number
		\item $12\pmod5=2$, move on to the next number
		\item $13\pmod5=3$, move on to the next number
		\item $14\pmod{11}=3$, move on to the next number
		\item $15\pmod{11}=4$, move on to the next number
		\item $16\pmod7=2$, move on to the next number
		\item $17\pmod5=2$, move on to the next number
		\item $18\pmod5=3$, move on to the next number
		\item $19\pmod7=7$, move on to the next number
		\item $20\pmod{11}=9$, move on to the next number
		\item $21$:
		\begin{itemize}
			\item $21\pmod2=1$
			\item $21\pmod3=0$
			\item $21\pmod5=1$
			\item $21\pmod7=0$
			\item $21\pmod{11}=-1$
		\end{itemize}
	\end{itemize}
	
	Therefore, $k=21$.

	\subsection*{Input Format}
	The first and only line of input contains $n$.

	\subsection*{Output Format}
	Your output should consist of one line that contains the number $k$.

	\subsection*{Constraints}
	$n\geq3$

	\subsection*{Sample Input/Output}
	\begin{tabular}{ |l|l| } 
		\hline
		\textbf{Input} & \textbf{Output} \\
		{\lstinputlisting{./testCases/input/input02.txt}} & {\lstinputlisting{./testCases/output/output02.txt}} \\ % use {\lstinputlisting{./testCases/input/input00.txt}} & {\lstinputlisting{./testCases/output/output00.txt}}
		\hline
	\end{tabular}


	\section*{Solution}
	Refer to the example given.



	\section*{Sample Program}
	\lstinputlisting[language=Python]{sampleSolution.py}
	

\end{document}