\documentclass[12pt]{report}
\usepackage[a4paper, total={7.3in, 9.7in}]{geometry}
\usepackage{amsmath}
\usepackage{upquote}
\usepackage{listings}
\usepackage{xcolor}
\usepackage{titlesec}
\usepackage{amssymb}

\definecolor{backgroundcolor}{rgb}{1, 1, 1}
\definecolor{commentstyle}{rgb}{0.365, 0.422, 0.475}
\definecolor{keywordstyle}{rgb}{0.6, 0.14, 0.576}
\definecolor{numberstyle}{rgb}{0.5, 0.5, 0.5}
\definecolor{stringstyle}{rgb}{0.77, 0.1, 0.08}

\lstdefinestyle{xcodecolor}{
    backgroundcolor=\color{backgroundcolor},   
    commentstyle=\color{commentstyle},
    keywordstyle=\color{keywordstyle},
    numberstyle=\scriptsize\color{numberstyle},
    stringstyle=\color{stringstyle},
    basicstyle=\ttfamily\footnotesize,
    breakatwhitespace=false,         
    breaklines=true,                 
    captionpos=b,                    
    keepspaces=true,                   
    numbersep=5pt,                  
    showspaces=false,                
    showstringspaces=false,
    showtabs=false,                  
    tabsize=2
}

\lstset{style=xcodecolor}

\usepackage[T1]{fontenc}
\usepackage{cascadia-code}

% Raised Rule Command:
%  Arg 1 (Optional) - How high to raise the rule
%  Arg 2            - Thickness of the rule
\newcommand{\raisedrule}[2][0em]{\leaders\hbox{\rule[#1]{1pt}{#2}}\hfill}

\setlength{\parindent}{0pt}
\titleformat{\section}
{\normalfont\Large\bfseries}{\thesection}{1em}{}[{\titlerule[0.8pt]}]

\begin{document}

	{\Large
	\textbf{To Trangles}}
	
	\vspace{0.4cm}
	DiPS CodeJam 23\raisedrule[0.25em]{1pt}
	\\
	% document

	\section*{Prompt}
	Pranav and Prithvi are back on their adventure (CodeJam 22, ``To the Treasure!'') this year. They find themselves at the beginning of a path that is dotted with obstacles, each of which as a similar puzzle: a grid of numbers is given, and they have to produce a ``triangle-like'' version of this grid. For example:
	$$\begin{bmatrix}
		1 & 2 & 3  \\
		4 & 5 & 6  \\
		7 & 8 & 9  \\
	\end{bmatrix}$$
	\begin{center}
		translates to
	\end{center}
	$$\begin{bmatrix}
	7 &   &   \\
	4 & 8 &   \\
	1 & 5 & 9 \\
	2 & 6 &   \\
	3 &   &   \\
 	\end{bmatrix}$$
	
	\subsection*{Input Format}
	\begin{itemize}
		\item The first line contains an integer $n$, denoting the size of the grid.
		\item The next $n$ lines contain $n$ space-separated numbers each.
	\end{itemize}

	\subsection*{Output Format}
	Your output should contain a space-separated triangle-shaped grid produced from the input.
	
	\subsection*{Sample Input/Output}
	\begin{tabular}{ |l|l| } 
		\hline
		\textbf{Input} & \textbf{Output} \\
		{\lstinputlisting{./testCases/input/input03.txt}} & {\lstinputlisting{./testCases/output/output03cd ...txt}} \\ % use {\lstinputlisting{./testCases/input/input00.txt}} & {\lstinputlisting{./testCases/output/output00.txt}}
		\hline
	\end{tabular}


	\section*{Sample Program}
	\lstinputlisting[language=Python]{sampleSolution.py}
	

\end{document}