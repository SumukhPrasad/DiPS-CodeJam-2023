\documentclass[12pt]{report}
\usepackage[a4paper, total={7.3in, 9.7in}]{geometry}
\usepackage{amsmath}
\usepackage{upquote}
\usepackage{listings}
\usepackage{xcolor}
\usepackage{titlesec}
\usepackage{amssymb}

\definecolor{backgroundcolor}{rgb}{1, 1, 1}
\definecolor{commentstyle}{rgb}{0.365, 0.422, 0.475}
\definecolor{keywordstyle}{rgb}{0.6, 0.14, 0.576}
\definecolor{numberstyle}{rgb}{0.5, 0.5, 0.5}
\definecolor{stringstyle}{rgb}{0.77, 0.1, 0.08}

\lstdefinestyle{xcodecolor}{
    backgroundcolor=\color{backgroundcolor},   
    commentstyle=\color{commentstyle},
    keywordstyle=\color{keywordstyle},
    numberstyle=\scriptsize\color{numberstyle},
    stringstyle=\color{stringstyle},
    basicstyle=\ttfamily\footnotesize,
    breakatwhitespace=false,         
    breaklines=true,                 
    captionpos=b,                    
    keepspaces=true,                   
    numbersep=5pt,                  
    showspaces=false,                
    showstringspaces=false,
    showtabs=false,                  
    tabsize=2
}

\lstset{style=xcodecolor}

\usepackage[T1]{fontenc}
\usepackage{cascadia-code}

% Raised Rule Command:
%  Arg 1 (Optional) - How high to raise the rule
%  Arg 2            - Thickness of the rule
\newcommand{\raisedrule}[2][0em]{\leaders\hbox{\rule[#1]{1pt}{#2}}\hfill}

\setlength{\parindent}{0pt}
\titleformat{\section}
{\normalfont\Large\bfseries}{\thesection}{1em}{}[{\titlerule[0.8pt]}]

\begin{document}

	{\Large
	\textbf{Wafer Cutting}}
	
	\vspace{0.4cm}
	DiPS CodeJam 23\raisedrule[0.25em]{1pt}
	\\
	% document

	\section*{Prompt}
	When manufacturing chips, circular silicon wafers are cut into dies of required size. The goal of this challenge is to maximise the number of whole dies of dimensions $l_x, l_y$ that can be cut from a wafer of a given diameter $d$. Your task is to find the maximum number of dies that can be cut from a wafer.
	
	The machine measures a distance $g_x$ from the left edge, and makes a vertical cut. Then it makes the needed number of additional vertical cuts, with spacing of the needed width. It makes the horizontal cuts in the same way: first one at a distance $g_y$, and then with spacing of height.

	The material remains perfectly still during the cutting: it's not allowed to move some slices of material before making the perpendicular cuts.

	The machine has a stepper motor that works at absolute accuracy but can only go integer-sized distances.


	\subsection*{Input Format}
	\begin{itemize}
		\item The first line of the input contains one integer $d$ that denotes the diameter of the wafer.
		\item The second line contains 2 space-separated integers $l_x$ and $l_y$ that denote the width and height of one die respectively.
	\end{itemize}

	\subsection*{Output Format}
	Your output should contain a single integer $w$ denoting the maximum number of dies that can be cut.

	\subsection*{Constraints}
	$0 < d \leq 450$

	\subsection*{Sample Input/Output}
	\begin{tabular}{ |l|l| } 
		\hline
		\textbf{Input} & \textbf{Output} \\
		{\lstinputlisting{./testCases/input/input00.txt}} & {\lstinputlisting{./testCases/output/output00.txt}} \\ % use {\lstinputlisting{./testCases/input/input00.txt}} & {\lstinputlisting{./testCases/output/output00.txt}}
		\hline
	\end{tabular}


	\section*{Solution}
	The challenge can be solved as follows:
	\begin{itemize}
		\item For each possible $g_x$ value:
		\begin{itemize}
			\item For each possible $g_y$ value:
			\begin{itemize}
				\item Assume the starting die count value is 0.
				\item For every die position, check if the die fits on the board.
				\item If the die fits, increment the die count.
			\end{itemize}
			\item If the current die count is greater than the greatest die count obtained previously, update the maximum value.
		\end{itemize}
	\end{itemize}


	\section*{Sample Program}
	\lstinputlisting[language=Python]{sampleSolution.py}
	

\end{document}